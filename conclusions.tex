\chapter{Conclusions}
\label{chap:concludions}

This PhD thesis presents one of the few computational-theoretical approaches to the memristor. By means of a bottom-up strategy we studied the prototypical TiO$_2$-based memristor from the raw materials to the device.

 The raw materials, oxygen-deficient phases of TiO$_2$, namely the corundum structure Ti$_2$O$_3$, the $\alpha-$, $\beta-$, and $\gamma-$Ti$_3$O$_5$, and the Magnéli phases Ti$_4$O$_7$ and Ti$_5$O$_9$ were studied in chapter \ref{chap:raw} with three flavors of DFT methods: a GGA functional, GGA$+U$, and hybrid functional HSE. These calculations revealed the presence of an intermediate band within the bandgap for most of these materials. The most important cases, the Magnéli phases, known to be formed inside the memristor after electroforming, presented a Ti(d) filled band lying close to CBM. Such band can act as a $n-$dopant to the surrounding TiO$_2$, thus changing its electronic properties. This was the first evidence of the importance of the electronic processes taking place inside memristors.

 The TiO$_2$-Ti$_4$O$_7$ interface spans the device, eventually connecting the opposing electrodes. Our band offset calculations in chapter \ref{chap:band-offset} showed the correct position of the intermediate band of the Magnéli phase with respect to the TiO$_2$ CBM. Lying just 160 meV below the TiO$_2$ unoccupied levels, this filled band can act as an electron donor to the TiO$_2$ matrix. This is a strong evidence of the electronic mechanisms being important for the memristive property. The separation of charges between the two structures suggested that a charge transfer might take place.

 Taking into consideration a possible charging and de-charging of the Magnéli phases formed inside the memristors, we studied the stability of these compounds using formation enthalpy calculations in chapter \ref{chap:charges}. The results show that Magnéli phases Ti$_4$O$_7$ and Ti$_5$O$_9$, as well as corundum phase Ti$_2$O$_3$ can be charged due to the presence of charge switching levels within the bandgap. These are exactly the intermediate bands reported in the previous studies. One important point is that when the intermediate band donates all its electrons, these oxygen-deficient phases become insulators. The arguments in favor of electronic switching are further reinforced by this result.

 Finally, a semiclassical picture of the memristor is presented in chapter \ref{chap:bandbend}. By considering a defective TiO$_2$ layer as the active part of the memristor, the Poisson equation was solved numerically by a self-consistent scheme. The fact that the defects inside the oxide could be charged and de-charged led to the fact that multiple solutions to the Poisson equation, \textit{i. e.}, multiple profiles of the potential energy barrier for electrons to cross the device, are possible. Using the calculated potential, the electronic current was calculated numerically and two distinct resistance states were obtained.

 A picture featuring all points studied in this thesis is presented in Figure \ref{fig:mem-concl}. There it is possible to observe all points covered in this project: the raw materials were investigated by DFT calculations of the electronic structure of the oxygen-deficient Ti$_n$O$_{2n-1}$ phases (chapter \ref{chap:raw})as well as the thermodynamic stability of these materials (chapter \ref{chap:charges}); the interface between TiO$_2$ and Ti$_4$O$_7$ was simulated for a better understanding of the band alignment between them (chapter \ref{chap:band-offset}); and finally the Poisson and Schr\"odinger equations were solved for the band bendings and electronic transport in the device (chapter \ref{chap:bandbend}).
 
Summarizing our conclusion is that electronic mechanisms similar to what was proposed by Simmons \cite{Simmons1967} definitely play an important role in resistance switching. As most of the literature focus on ionic drift models, the electronic models do not receive the same attention, but there is no reason to consider both models exclusive. The defects are present in all memristors made of a variety of materials, and in many situations these defects can become charged and affect the electronic properties of the active layer of the device. Of course these models do not exclude each other: there may be a contribution from both the charge and de-charge of defects as well as the migration of ions, but to understand the interplay between these mechanisms one should couple the ionic motion into the Poisson solver. One last remark is that the ionic motion is probably the dominant process during the electroforming step, when the defect concentration increases and thus enables the charging and de-charging of the active layer.

\begin{center}
  \begin{figure}[h!]
    \begin{center}
      \includegraphics[width=0.7\textwidth]{img/memristor-concl-B.jpg}    
      \caption{A bottom-up approach to model the memristor. From the raw materials (Ti$_n$O$_{2n-1}$ phases were studied in chapters \ref{chap:raw} and \ref{chap:charges}), going through the interfaces (band alignment between TiO$_2$ and Ti$_4$O$_7$, chapter \ref{chap:band-offset}), and finally the simulation of the device (solution of Poisson and Schr\"odinger equations, chapter \ref{chap:bandbend}).} 
      \label{fig:mem-concl} 
    \end{center}
  \end{figure}
\end{center}
%The two main perspectives of this work are: i) study of the ionic motion within the active layer. Using minimum-path methods such as the nudged elastic band (NEB) \cite{Henkelman2000a,Henkelman2000b} we want to evaluate the real barriers for ionic motion in the memristive materials. The effects of electric fields and temperature must be taken into account for a correct description of the system \cite{Crehuet2003}. ii) more models for electronic transport in defective semiconductors such as the variable range hopping (VRH) \cite{}. The whole picture should be available when both models for electronic switching and their simultaneous effects are understood in an atomic level.