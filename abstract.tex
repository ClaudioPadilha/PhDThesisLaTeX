%%%%%%%%%%%%%%%%%%%%%%%%%%%%%%%%%%%%%%%%%%%%%%%%%%%%%%%%%%%%%%%%%%%%%%%%%%%%%%%%%%%%

\newpage
%\thispagestyle{empty}
%\frontmatter
%==================================================================================%
\centerline{\large \textbf{Abstract}}

%\vspace{0.8cm}

\begin{quote}
The resistive switching or memristive property is the ability of a material to change its electrical resistance due to the application of an electric field. The memristor is a two-terminal device with this property that is capable of storing information as its resistance state, being architectured in a metal/insulator/metal stacking. This device can revolutionize the memory industry by providing fast switching and large retention times as well as high-density capabilities. However, its working principle is not completely understood at an atomic level, thus its application as next-generation resistive memories is hindered. Two mechanisms are proposed: ion drift mechanisms claim that the electric field and temperature gradients inside the device can form and dissolve a conducting filament, changing the electrical resistivity. On the other hand, electronic models consider charge trapping and de-trapping inside the insulator layer as the cause of the resistivity change. In this work we use a heuristic computational approach---density functional theory calculations and other numerical solutions---to understand the processes developing at the atomic scale inside TiO$_2$-based devices. Our results show that the oxygen deficiency in this material leads to the formation of a series of phases Ti$_n$O$_{2n-1}$ that present an intermediate band which can become charged when properly interfaced. The self-consistent-numerical solver of the Poisson equation shows multiple solutions that are related to the resistance states, and finally the potential is used in a transmission code that results in theoretical $i \times V$ curves for the memristor. \\ \textbf{Keywords:} Memristor, Memristive Devices, Density Functional Theory, DFT, Computational Simulation, Titanium Oxide, Magnéli Phases.
\end{quote}

\newpage

%%%%%%%%%%%%%%%%%%%%%%%%%%%%%%%%%%%%%%%%%%%%%%%%%%%%%%%%%%%%%%%%%%%%%%%%%%%%%%%%%%%%
\centerline{\large \textbf{Resumo}}

%\vspace{0.8cm}

\begin{quote}
A propriedade de chaveamento da resitência ou memoristiva é a habilidade de um material de alterar seu estado de resistência elétrica devido a um campo elétrico. O memoristor é um dispositivo de dois terminais com tal propriedade capaz de armazenar informação através de sua resistência, constituído de uma estrutura metal/isolante/metal. Este dispositivo pode revolucionar a indústria de memórias por apresentar tempos de chaveamento rápidos e de retenção longos, assim como altas densidades. Entretanto, seu princípio de funcionamento não é totalmente entendido a nível atômico, logo sua aplicação é impedida. Dois mecanismos são propostos: o mecanismo de difusão-deriva de íons afirma que campos elétricos e gradientes de temperatura formam e dissolvem canais condutores, alterando a resistividade. Por outro lado, modelos eletrônicos consideram o aprisionamento e liberação de cargas como causa da mudança da resistividade. Neste trabalho utilizamos uma abordagem heurística---cálculos de teoria do funcional da densidade e soluções numéricas---para entender os processos ocorrendo em escala atômica no interior de dispositivos baseados em TiO$_2$. Os resultados mostram que a dificência em oxigênio neste caso leva à formação de fases Ti$_n$O$_{2n-1}$ que apresentam uma banda intermediária, a qual pode se tornar carregada quando propriamente interfaceada. A resolução numérica da equação de Poisson apresenta múltiplas soluções relacionadas a diferentes estados de resistência, estas soluções são usadas em um código de transmissão que fornece curvas teóricas $i \times V$ para o memoristor. \\ \textbf{Palavras-chave:} Memoristores, Dispositivos Memoristivos, Teoria do Funcional da Densidade, DFT, Simulações Computacionais, Óxido de Titânio, Fases Magnéli.
\end{quote}

\newpage